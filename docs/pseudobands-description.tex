\documentclass[hyperref, a4paper]{article}

\usepackage{textgreek}
\usepackage{geometry}
\usepackage{titling}
\usepackage{titlesec}
\usepackage{footnote}
\usepackage[colorinlistoftodos]{todonotes}
\usepackage{booktabs}
\usepackage{array}
\usepackage{multirow}
\usepackage{amsmath, amssymb, amsthm}
\usepackage{mathtools}
\usepackage{bbm}
\usepackage{graphicx}
\usepackage{subcaption}
\usepackage{physics}
\usepackage{tensor}
\usepackage{siunitx}
\usepackage[version=4]{mhchem}
\usepackage{tikz}
\usepackage{xcolor}
\usepackage{listings}
\usepackage{autobreak}
\usepackage[colorlinks,unicode]{hyperref} % , linkcolor=black, anchorcolor=black, citecolor=black, urlcolor=black, filecolor=black
\usepackage{xurl}
\usepackage[most]{tcolorbox}
\usepackage[backend=bibtex,sorting=none,doi=false,isbn=false,url=false]{biblatex}
\addbibresource{gw-bse.bib}

\usepackage{prettyref}

% Page style; to be removed when this article is placed in another template
\geometry{left=3.18cm,right=3.18cm,top=2.54cm,bottom=2.54cm}
\titlespacing{\paragraph}{0pt}{1pt}{10pt}[20pt]
\setlength{\droptitle}{-5em}


% Math operators
\DeclareMathOperator{\timeorder}{\mathcal{T}}
\DeclareMathOperator{\diag}{diag}
\DeclareMathOperator{\legpoly}{P}
\DeclareMathOperator{\primevalue}{P}
\DeclareMathOperator{\sgn}{sgn}
\newcommand*{\ii}{\mathrm{i}}
\newcommand*{\ee}{\mathrm{e}}
\newcommand*{\const}{\mathrm{const}}
\newcommand*{\suchthat}{\quad \text{s.t.} \quad}
\newcommand*{\argmin}{\arg\min}
\newcommand*{\argmax}{\arg\max}
\newcommand*{\normalorder}[1]{: #1 :}
\newcommand*{\pair}[1]{\langle #1 \rangle}
\newcommand*{\fd}[1]{\mathcal{D} #1}
\DeclareMathOperator{\bigO}{\mathcal{O}}


% Embedded codes
\lstdefinestyle{console}{
    basicstyle=\footnotesize\ttfamily,
    breaklines=true,
    postbreak=\mbox{\textcolor{red}{$\hookrightarrow$}\space}
}

% Reference formatting
\newrefformat{fig}{Fig.~\ref{#1}}
\newrefformat{tbl}{Table~\ref{#1}}

% TiKZ settings
\usetikzlibrary{calc}
\tikzset{every picture/.style={line width=0.3pt}} 

% Displaying chemical formula in bookmarkers

\pdfstringdefDisableCommands{%
  \def\\{}%
  \def\ce#1{<#1>}%
}

\pdfstringdefDisableCommands{%
  \def\texttt#1{<#1>}%
  \def\mathbb#1{#1}%
}
\pdfstringdefDisableCommands{\def\eqref#1{(\ref{#1})}}

\makeatletter
\pdfstringdefDisableCommands{\let\HyPsd@CatcodeWarning\@gobble}
\makeatother



\newcommand{\citetime}[1]{item \ref{#1} in \prettyref{sec:timeline}}
\newcommand{\address}[1]{\href{#1}{\url{#1}}}
\newcommand{\shortcode}[1]{\texttt{#1}}

\DeclareSIUnit{\au}{a.u.}

\lstset{style = console}

% Labels 
\newcommand{\strainplot}{\prettyref{fig:one-shot}(d)}
\newcommand{\strainband}{\prettyref{fig:one-shot}(f)}
\newcommand{\nostrainband}{\prettyref{fig:one-shot}(e)}
\newcommand{\overlap}{\prettyref{fig:overlap}}

\title{Supplementary material}
\author{Bowen Hou, Jinyuan Wu, Xingzhi Sun, Smita Krishnaswamy, and Diana Y. Qiu}

\begin{document}

\maketitle

\section{Pseudobands and the physics behind our model}

A $GW$ calculation under the generalized plasmon-pole (GPP) model
involves the calculation of the zero-frequency dielectric matrix $\chi_{\vb{G} \vb{G}'}(\vb{q}, \omega = 0)$
(done in \shortcode{epsilon} in BerkeleyGW)
and the calculation of the screened-exchange (SX) and Coulomb hole (CH) part 
of the single-electron self-energy 
(done in \shortcode{sigma} in BerkeleyGW) \cite{deslippe2012berkeleygw}.
Calculation of $\chi$ and $\Sigma^{\text{CH}}$ involves 
summation over theoretically infinite bands, 
which is a bottleneck of the performance of the procedure.
The problem is usually handled by the pseudobands technique \cite{del2019large},
in which the empty bands are divided into one low-energy protected subspace 
and a series of pseudobands blocks; 
each of the blocks contain bands with comparable energies.
The energies of each block are then replaced by a single energy, 
which is the average of the former, 
and for each $\vb{k}$ point, 
all $\ket*{n \vb{k}}$ states in the block $S$ is replaced by $\sum_n^S \ket*{n \vb{k}}$.
In this way each pseudobands block is replaced by a flat band, 
and at each $\vb{k}$ point we have an effective, unnormalized wave function;
this band is known as a pseudoband. 
The protected bands and the pseudobands are then fed to \shortcode{epsilon} and \shortcode{sigma}.
In this section, I briefly discuss why this seemingly crude approximation works.

The expression of the zero-frequency dielectric matrix is 
\begin{equation}
    \chi_{\vb{G} \vb{G}'}(\vb{q}, \omega = 0)
    = \sum_{\vb{k}} \sum_{n}^{\text{occ}} \sum_{n'}^{\text{emp}} 
    M_{n n'} (\vb{k}, \vb{q}, \vb{G}) M^*_{nn'} (\vb{k}, \vb{q}, \vb{G}') 
    \frac{
        2
    }{
        E_{n \vb{k} + \vb{q}} - E_{n' \vb{k}} 
    },
\end{equation}
where 
\begin{equation}
    M_{nn'}(\vb{k}, \vb{q}, \vb{G}) = \mel*{n \vb{k} + \vb{q}}{\ee^{\ii (\vb{q} + \vb{G}) \cdot \vb{r}}}{n' \vb{k}}.
\end{equation}
When pseudobands is used,
the high-energy terms in $\chi_{\vb{G} \vb{G}'}(\vb{q}, \omega = 0)$ are therefore replaced by  
(P.B. means pseudobands)
\begin{equation}
    \chi^{\text{P.B. terms}}_{\vb{G} \vb{G}'}(\vb{q}, \omega = 0)
    = \sum_{\vb{k}}  \sum_{S}^{\text{P.B. blocks}} \frac{
        2
    }{
        E_{n \vb{k} + \vb{q}} - \bar{E}_S 
    } 
    \sum_{n_1', n_2'}^{S} \sum_{n}^{\text{occ}}
    M_{n n'_1} (\vb{k}, \vb{q}, \vb{G}) M^*_{nn'_2} (\vb{k}, \vb{q}, \vb{G}') .
    \label{eq:chi-pseudobands}
\end{equation}
To justify the pseudobands technique, 
it is sufficient to show that the unwanted $n_1' \neq n_2'$ cross terms 
that do not appear in the definition of $\chi$ vanish after the summations over $\vb{k}$ and $n$.

Because the energies of states in each pseudobands block are comparable, 
in each pseudobands block, the states share the same set of predominant $\vb{G}$ vectors.
Thus the states in pseudobands block $S$ can be written as 
\begin{equation}
    \braket*{\vb{r}}{n' \vb{k}} = \frac{1}{\sqrt{V}} \sum_i^N 
    c_{n' \vb{k}}(\vb{G}^{(i)}_{S \vb{k}}) \ee^{\ii (\vb{k} + \vb{G}^{(i)}_{S \vb{k}}) \cdot \vb{r}}, \quad 
    \sum_{i}^{N} c_{n'_1 \vb{k}}(\vb{G}^{(i)}_{S \vb{k}}) c_{n'_2 \vb{k}}^* (\vb{G}^{(i)}_{S \vb{k}}) = \delta_{n'_1 n'_2},
\end{equation}
and therefore 
\begin{equation}
    M_{nn'}(\vb{k}, \vb{q}, \vb{G}) = \sum_{i}^{N} c_{n' \vb{k}}(\vb{G}^{(i)}_{S \vb{k}})  c_{n \vb{k} + \vb{q}}^*(\vb{G} + \vb{G}_{S \vb{k}}^{(i)}).
\end{equation}
The pseudobands approximation of the diagonal $\chi_{\vb{G} \vb{G}}$ components of the polarizability 
is proportional to 
\begin{equation}
    \begin{aligned}
    &\quad \sum_n^{\text{occ}} M_{n n'_1} (\vb{k}, \vb{q}, \vb{G}) M^*_{n n'_2} (\vb{k}, \vb{q}, \vb{G}) \\
    &= \sum_{i , j}^N 
    c_{n'_1 \vb{k}}(\vb{G}_{S \vb{k}}^{(i)}) 
    c^{*}_{n'_2 \vb{k}}(\vb{G}_{S \vb{k}}^{(j)})
    \sum_{n}^{\text{occ}} 
    c_{n \vb{k} + \vb{q}}^*(\vb{G} + \vb{G}^{(i)}_{S \vb{k}})
    c_{n \vb{k} + \vb{q}}  (\vb{G} + \vb{G}^{(j)}_{S \vb{k}}) \\
    &\propto \sum_{i , j}^N  
    c_{n'_1 \vb{k}}(\vb{G}^{(i)}_{S \vb{k}}) 
    c^{*}_{n'_2 \vb{k}}(\vb{G}^{(j)}_{S \vb{k}}) \delta_{ij} = \delta_{n_1' n_2'}.
    \end{aligned}
\end{equation}
In the third step we argue that the summation over the occupied states leads to 
an approximate, non-normalized orthogonal relation,
because the subspace spanned by the dominant $\vb{G}$ components of the occupied states 
is expected to largely overlap with the subspace of the occupied states,
and therefore although for $i=j$ terms, the summation over occupied $n$ is usually far less than one, 
the fast oscillation of $i \neq j$ terms when $n$ changes quickly brings the summation to zero.
Therefore, the unwanted non-diagonal terms in \eqref{eq:chi-pseudobands} 
can be ignored when $\vb{G} = \vb{G}'$.

For the $\vb{G} \neq \vb{G}'$ terms we similarly have 
\begin{equation}
    \begin{aligned}
        &\quad \sum_n^{\text{occ}} M_{n n'_1} (\vb{k}, \vb{q}, \vb{G}) M^*_{n n'_2} (\vb{k}, \vb{q}, \vb{G}') \\
        &= \sum_{i , j}^N 
        c_{n'_1 \vb{k}}(\vb{G}_{S \vb{k}}^{(i)}) 
        c^{*}_{n'_2 \vb{k}}(\vb{G}_{S \vb{k}}^{(j)})
        \sum_{n}^{\text{occ}} 
        c_{n \vb{k} + \vb{q}}^*(\vb{G}  + \vb{G}^{(i)}_{S \vb{k}})
        c_{n \vb{k} + \vb{q}}  (\vb{G}' + \vb{G}^{(j)}_{S \vb{k}}) \\
        &\propto \sum_{i , j}^N  
        c_{n'_1 \vb{k}}(\vb{G}^{(i)}_{S \vb{k}}) 
        c^{*}_{n'_2 \vb{k}}(\vb{G}^{(i)}_{S \vb{k}} + \vb{G} - \vb{G}') .
    \end{aligned}
\end{equation}
Numerical experiments reveal that the unwanted $n_1' \neq n_2'$ terms in the  
$\vb{G} \neq \vb{G}'$ components in the above equation 
do not cancel each other; 
the summation over $\vb{k}$ however could eliminate these terms.
We note that the $1 / (E_{\text{v}} - E_{\text{c}})$ factor in \eqref{eq:chi-pseudobands} 
is approximately a constant as $\vb{k}$ varies, 
and therefore the summation over $\vb{k}$ in \eqref{eq:chi-pseudobands} 
contains the following factor
\begin{equation}
    \sum_{\vb{k}} c_{n_1' \vb{k}}(\vb{G}^{(i)}) c^*_{n_2' \vb{k}}(\vb{G}^{(j)})
    \propto \sum_{\vb{k}} \ee^{\ii \theta_{n_1' \vb{k}} - \ii \theta_{n_2' \vb{k}}}
    \stackrel{N_{\vb{k}} \to \infty}{\longrightarrow} \delta_{n_1' n_2'},
    \label{eq:convergence}
\end{equation}
where $\ee^{\ii \theta_{n \vb{k}}}$ is the random global phase factor 
introduced in the diagonalization process,
which changes much more rapidly than the rest part of $c_{n \vb{k}}$
as $\vb{k}$ runs over the Brillouin zone sampling.
In conclusion, the validity of the pseudobands technique for high-energy bands
is well-justified for the whole $\chi_{\vb{G} \vb{G}'}(\vb{q}, \omega = 0)$
because of the summation over $\vb{k}$ and/or $n$.

The analysis of pseudobands in \shortcode{sigma} is more complicated
because in the Coulomb hole part of the GPP self-energy
%\begin{equation}
%    \begin{aligned}
%        \mel*{n \vb{k}}{\Sigma_{\mathrm{SX}}(E)}{n' \vb{k}} &=
%        -\sum_{n^{\prime \prime}}^{\mathrm{occ}} \sum_{\vb{q} \vb{G} \vb{G}'} M_{n^{\prime \prime} n}^*(\vb{k},-\vb{q},-\vb{G}) M_{n^{\prime \prime} n^{\prime}}\left(\vb{k},-\vb{q},-\vb{G}^{\prime}\right) \\
%        & \times\left[\epsilon_{\vb{G G}^{\prime}}\right]^{-1}\left(\vb{q} ; E-E_{n^{\prime \prime} \vb{k}-\vb{q}}\right) v\left(\vb{q}+\vb{G}^{\prime}\right)
%        \end{aligned}
%\end{equation}
%and 
\begin{equation}
    \begin{aligned}
        \left\langle n \mathbf{k}\left|\Sigma_{\text{C H}}(E)\right| n^{\prime} \mathbf{k}\right\rangle=\frac{1}{2} & \sum_{n^{\prime \prime}} \sum_{\mathbf{q G G}^{\prime}} M_{n^{\prime \prime} n}^*(\mathbf{k},-\mathbf{q},-\mathbf{G}) M_{n^{\prime \prime} n^{\prime}}\left(\mathbf{k},-\mathbf{q},-\mathbf{G}^{\prime}\right) \\
        & \times \frac{\Omega_{\mathbf{G G}^{\prime}}^2(\mathbf{q})\left(1- \ii \tan \phi_{\mathbf{G G}^{\prime}}(\mathbf{q})\right)}{\tilde{\omega}_{\mathbf{G G}^{\prime}}(\mathbf{q})\left(E-E_{n^{\prime \prime} \mathbf{k}-\mathbf{q}}-\tilde{\omega}_{\mathbf{G G}^{\prime}}(\mathbf{q})\right)} v\left(\mathbf{q}+\mathbf{G}^{\prime}\right),
    \end{aligned}
    \label{eq:sigma-ch}
\end{equation}
where $\Omega_{\vb{G} \vb{G}'}$, $\tilde{\omega}_{\vb{G} \vb{G}'}$ and $\phi_{\vb{G} \vb{G}'}$
are quantities calculated from $\chi_{\vb{G} \vb{G}'}(\vb{q}, \omega = 0)$ 
and the momentum space ground state density $\rho(\vb{G})$.
The right hand side contains complicated dependence on $\vb{q}$, $\vb{G}$ and $\vb{G}'$.
However, note that when $n''$ is high and is within one of the pseudobands blocks,
typically the magnitudes of $\vb{G}$ and $\vb{G}'$ are large enough 
to dominate $v(\vb{q} + \vb{G}) = 4 \pi / \abs*{\vb{q} + \vb{G}}^2$, 
and therefore the rapid variation of the random phase factor in $\ket*{n'' \vb{k} - \vb{q}}$ 
compared with the relatively slow variance of second line of \eqref{eq:sigma-ch}
as $\vb{q}$ runs over the Brillouin zone sampling
again provides us with the opportunity to justify using pseudobands here:
we have 
\begin{equation}
    \mel*{n \vb{k}}{\Sigma_{\text{CH}}^{\text{high energy bands}}(E)}{n' \vb{k}} \propto 
    \sum_{\vb{G}, \vb{G}'} \sum_{n''}^{\text{high energy bands}} \sum_{\vb{q}} M^*_{n'' n}(\vb{k}, - \vb{q}, -\vb{G})
    M_{n'' n'}(\vb{k}, - \vb{q}, - \vb{G}'),
    \label{eq:high-energy-sigma-ch}
\end{equation}
and by applying the arguments in \eqref{eq:convergence}, 
the pseudobands estimation of the high-energy terms in $\Sigma_{\text{CH}}$ is 
\begin{equation}
    \begin{aligned}
        \mel*{n \vb{k}}{\Sigma_{\text{CH}}^{\text{P.B. terms}}(E)}{n' \vb{k}} &\propto 
        \sum_{\vb{G}, \vb{G}'} \sum_{S}^{\text{P.B. blocks}} \sum_{n''_1, n''_2}^{S} \sum_{\vb{q}}
        M^*_{n''_1 n}(\vb{k}, - \vb{q}, -\vb{G})
        M_{n''_2 n'}(\vb{k}, - \vb{q}, - \vb{G}') \\
        &\propto \sum_{\vb{q}} \ket*{n_1'' \vb{k} - \vb{q}} \bra*{n_2'' \vb{k} - \vb{q}}
        \stackrel{N_{\vb{q}} \to \infty}{\propto} \delta_{n_1'' n_2''},
    \end{aligned}
\end{equation}
and the unwanted $n_1'' \neq n_2''$ cross terms that do not appear in \eqref{eq:high-energy-sigma-ch} again vanish.

The aforementioned fact that the random phase factor in DFT diagonalization 
ensures the pseudobands technique can be further exploited by 
\emph{intentionally} inserting random phase factors before $\ket*{n \vb{k}}$ states 
for a given $\vb{k}$ point
and replacing a pseudobands block by several flat unnormalized bands, 
instead of just one such band;
under such a scheme the size of pseudobands blocks can be drastically increased, 
further improving the speed of the $GW$ methodology;
specifically, an empirical observation is that 
the valence bands can be pseudo-ized as well,
and the protected subspace can be very small, 
without any real damage to accuracy (TODO: cite Altman et al. forthcoming).

The above discussion shows that there exists a relatively simple and smooth mapping
from the protected bands, ground state density and pseudobands to the $GW$ energies.
Even after the pseudobands procedure, 
there are still hundred of (protected and pseudo) bands;
in our model, we tentatively eliminate the protected subspace 
and pseudo-ize all bands, 
and also reduce the number of pseudobands subspaces to (TODO: 5??) for each material,
hoping that given that the ground state electron density 
in principle contains all information of the material,
as shown by the foundation of DFT \cite{hohenberg1964inhomogeneous},
having the ground state electron density somehow compensates for the information loss.

\printbibliography

\end{document}
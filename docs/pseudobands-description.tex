\documentclass[hyperref, a4paper]{article}

\usepackage{textgreek}
\usepackage{geometry}
\usepackage{titling}
\usepackage{titlesec}
\usepackage{footnote}
\usepackage[colorinlistoftodos]{todonotes}
\usepackage{booktabs}
\usepackage{array}
\usepackage{multirow}
\usepackage{amsmath, amssymb, amsthm}
\usepackage{mathtools}
\usepackage{bbm}
\usepackage{graphicx}
\usepackage{subcaption}
\usepackage{physics}
\usepackage{tensor}
\usepackage{siunitx}
\usepackage[version=4]{mhchem}
\usepackage{tikz}
\usepackage{xcolor}
\usepackage{listings}
\usepackage{autobreak}
\usepackage[colorlinks,unicode]{hyperref} % , linkcolor=black, anchorcolor=black, citecolor=black, urlcolor=black, filecolor=black
\usepackage{xurl}
\usepackage[most]{tcolorbox}
\usepackage[backend=bibtex,sorting=none,doi=false,isbn=false,url=false]{biblatex}
\addbibresource{monolayer-WTe2.bib}
\addbibresource{methods.bib}

\usepackage{prettyref}

% Page style; to be removed when this article is placed in another template
\geometry{left=3.18cm,right=3.18cm,top=2.54cm,bottom=2.54cm}
\titlespacing{\paragraph}{0pt}{1pt}{10pt}[20pt]
\setlength{\droptitle}{-5em}


% Math operators
\DeclareMathOperator{\timeorder}{\mathcal{T}}
\DeclareMathOperator{\diag}{diag}
\DeclareMathOperator{\legpoly}{P}
\DeclareMathOperator{\primevalue}{P}
\DeclareMathOperator{\sgn}{sgn}
\newcommand*{\ii}{\mathrm{i}}
\newcommand*{\ee}{\mathrm{e}}
\newcommand*{\const}{\mathrm{const}}
\newcommand*{\suchthat}{\quad \text{s.t.} \quad}
\newcommand*{\argmin}{\arg\min}
\newcommand*{\argmax}{\arg\max}
\newcommand*{\normalorder}[1]{: #1 :}
\newcommand*{\pair}[1]{\langle #1 \rangle}
\newcommand*{\fd}[1]{\mathcal{D} #1}
\DeclareMathOperator{\bigO}{\mathcal{O}}


% Embedded codes
\lstdefinestyle{console}{
    basicstyle=\footnotesize\ttfamily,
    breaklines=true,
    postbreak=\mbox{\textcolor{red}{$\hookrightarrow$}\space}
}

% Reference formatting
\newrefformat{fig}{Fig.~\ref{#1}}
\newrefformat{tbl}{Table~\ref{#1}}

% TiKZ settings
\usetikzlibrary{calc}
\tikzset{every picture/.style={line width=0.3pt}} 

% Displaying chemical formula in bookmarkers

\pdfstringdefDisableCommands{%
  \def\\{}%
  \def\ce#1{<#1>}%
}

\pdfstringdefDisableCommands{%
  \def\texttt#1{<#1>}%
  \def\mathbb#1{#1}%
}
\pdfstringdefDisableCommands{\def\eqref#1{(\ref{#1})}}

\makeatletter
\pdfstringdefDisableCommands{\let\HyPsd@CatcodeWarning\@gobble}
\makeatother



\newcommand{\citetime}[1]{item \ref{#1} in \prettyref{sec:timeline}}
\newcommand{\address}[1]{\href{#1}{\url{#1}}}
\newcommand{\shortcode}[1]{\texttt{#1}}

\DeclareSIUnit{\au}{a.u.}

\lstset{style = console}

% Labels 
\newcommand{\strainplot}{\prettyref{fig:one-shot}(d)}
\newcommand{\strainband}{\prettyref{fig:one-shot}(f)}
\newcommand{\nostrainband}{\prettyref{fig:one-shot}(e)}
\newcommand{\overlap}{\prettyref{fig:overlap}}

\title{Supplementary material}
\author{Bowen Hou, Jinyuan Wu, Xingzhi Sun, Smita Krishnaswamy, and Diana Y. Qiu}

\begin{document}

\maketitle

\begin{equation}
    M_{nn'}(\vb{k}, \vb{q}, \vb{G}) = \mel*{n \vb{k} + \vb{q}}{\ee^{\ii (\vb{q} + \vb{G}) \cdot \vb{r}}}{n' \vb{k}}
\end{equation}

\begin{equation}
    \chi_{\vb{G} \vb{G}'}(\vb{q}, \omega = 0)
    = \sum_{\vb{k}} \sum_{n}^{\text{occ}} \sum_{n'}^{\text{emp}} 
    M_{n n'} (\vb{k}, \vb{q}, \vb{G}) M^*_{nn'} (\vb{k}, \vb{q}, \vb{G}') 
    \frac{
        2
    }{
        E_{n \vb{k} + \vb{q}} - E_{n' \vb{k}} 
    }
\end{equation}

In the conventional pseudobands technique, 
the empty states are divided into one low-energy protected subspace 
and a series of pseudobands blocks; 
each of the blocks contain bands with comparable energies.

\begin{equation}
    \chi^{\text{P.B. blocks}}_{\vb{G} \vb{G}'}(\vb{q}, \omega = 0)
    = \sum_{\vb{k}}  \sum_{S}^{\text{P.B. terms}} \frac{
        2
    }{
        E_{n \vb{k} + \vb{q}} - \bar{E}_S 
    } 
    \sum_{n_1', n_2'}^{S} \sum_{n}^{\text{occ}}
    M_{n n'_1} (\vb{k}, \vb{q}, \vb{G}) M^*_{nn'_2} (\vb{k}, \vb{q}, \vb{G}') 
\end{equation}

In each block, we may see states containing a primary $\vb{G}$ component, 
\begin{equation}
    \braket*{\vb{r}}{n' \vb{k}} = \ee^{\ii \theta_{n' \vb{k}}} \ee^{\ii (\vb{k} + \vb{G}_{n'}) \cdot \vb{r}}, 
\end{equation}
or $N$ degenerate states containing $N$ primary $\vb{G}$ components
\begin{equation}
    \braket*{\vb{r}}{n' \vb{k}} = \sum_i c_i \ee^{\ii (\vb{k} + \vb{G}_i) \cdot \vb{r}}.
\end{equation}
The latter case is due to symmetry.

When $n_1', n_2'$ both contain only one primary $\vb{G}$ component, and therefore
\begin{equation}
    M_{nn'}(\vb{k}, \vb{q}, \vb{G}) = \ee^{\ii \theta_{n' \vb{k}}} c^*_{n \vb{k} + \vb{q}}(\vb{G} + \vb{G}_{n'}).
\end{equation}
their contribution to $\chi$ is proportional to 
\begin{equation}
    \sum_n^{\text{occ}} M_{n n'_1} (\vb{k}, \vb{q}, \vb{G}) M^*_{n n'_2} (\vb{k}, \vb{q}, \vb{G}) 
    = \sum_n^{\text{occ}} c^*_{n \vb{k} + \vb{q}}(\vb{G} + \vb{G}_{n'_1}) c_{n \vb{k} + \vb{q}}(\vb{G} + \vb{G}_{n'_2})
    \approx \delta_{n_1' n_2'}.
    \label{eq:approx-completeness}
\end{equation}
The right hand side likely becomes $\delta_{n_1' n_2'}$ after the summation, 
because the subspace spanned by the dominant $\vb{G}$ components of the occupied states 
is expected to largely overlap with the subspace of the occupied states.

\begin{equation}
    \braket*{\vb{r}}{n' \vb{k}} = \frac{1}{\sqrt{N}} \sum_i^N 
    c_{n' \vb{k}}^{(i)} \ee^{\ii (\vb{k} + \vb{G}^{(i)}_{n' \vb{k}}) \cdot \vb{r}}, \quad 
    \abs*{c_{n' \vb{k}}^{(i)}}^2 = 1, \quad 
    \sum_{i}^{N} c^{(i)}_{n'_1 \vb{k}} c^{(i)*}_{n'_2 \vb{k}} = N \delta_{n'_1 n'_2}.
\end{equation}
Here since we assume $\ket*{n_1' \vb{k}}$ and $n_2' \vb{k}$ 
share the same set of predominant $\vb{G}$ vectors, 
we replace $\vb{G}^{(i)}_{n'_{1, 2} \vb{k}}$ by $\vb{G}^{(i)}_{n' \vb{k}}$
(and the order of the $\vb{G}$ vectors are also set the same for $n_1'$ and $n_2'$), 
and thus 
\begin{equation}
    \begin{aligned}
    &\quad \sum_{n_1'n _2'}^S \sum_n^{\text{occ}} M_{n n'_1} (\vb{k}, \vb{q}, \vb{G}) M^*_{n n'_2} (\vb{k}, \vb{q}, \vb{G}) \\
    &= \frac{1}{N} \sum_{i, j}^N \sum_{n}^{\text{occ}} 
    c^{(i)}_{n'_1 \vb{k}} c^{(j) *}_{n'_2 \vb{k}}
    c_{n \vb{k} + \vb{q}}^*(\vb{G} + \vb{G}^{(i)}_{n' \vb{k}})
    c_{n \vb{k} + \vb{q}}  (\vb{G} + \vb{G}^{(j)}_{n' \vb{k}}) .
    \end{aligned}
\end{equation}
The $i = j$ terms evaluate as 
\begin{equation}
    \sum_{i}^N \sum_{n}^{\text{occ}} 
    c^{(i)}_{n'_1 \vb{k}} c^{(i) *}_{n'_2 \vb{k}}
    \abs{c_{n \vb{k} + \vb{q}}^*(\vb{G} + \vb{G}^{(i)}_{n' \vb{k}})}^2 \propto \delta_{n_1' n_2'}.
\end{equation}
Note that the $\vb{G}^{(i)}_{n' \vb{k}}$ components are connected by symmetry operations, 
and we have 
\[
    \abs{c_{n \vb{k} + \vb{q}}^*(\vb{G} + \vb{G}^{(i)}_{n' \vb{k}})}^2 = \const
\]
for $1 \leq i \leq N$, 
and therefore the above equation is just a constant times 
the summation of $c^{(i)}_{n'_1 \vb{k}} c^{(i) *}_{n'_2 \vb{k}}$ over $i$, 
which then results in $\delta_{n_1' n_2'}$.
The $i \neq j$ terms evaluate as 
\[
    \sum_{i \neq j}^N 
    c^{(i)}_{n'_1 \vb{k}} c^{(j) *}_{n'_2 \vb{k}}
    \sum_{n}^{\text{occ}} 
    c_{n \vb{k} + \vb{q}}^*(\vb{G} + \vb{G}^{(i)}_{n' \vb{k}})
    c_{n \vb{k} + \vb{q}}  (\vb{G} + \vb{G}^{(j)}_{n' \vb{k}}) 
    \approx \sum_{i \neq j}^N  c^{(i)}_{n'_1 \vb{k}} c^{(j) *}_{n'_2 \vb{k}} \delta_{ij} = 0.
\]
In the first step we again use the argument used to prove \eqref{eq:approx-completeness}.


Numerical TODO list: 
\begin{itemize}
    \item $\chi_{\vb{G} \vb{G}'}$: diagonal, or not? Yes, it seems.
    \item Do different bands contain different $\vb{G}$ vectors?
\end{itemize}

\end{document}
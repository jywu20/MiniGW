\documentclass[hyperref, a4paper]{article}

\usepackage{textgreek}
\usepackage{geometry}
\usepackage{titling}
\usepackage{titlesec}
\usepackage{footnote}
\usepackage[colorinlistoftodos]{todonotes}
\usepackage{booktabs}
\usepackage{array}
\usepackage{multirow}
\usepackage{amsmath, amssymb, amsthm}
\usepackage{mathtools}
\usepackage{bbm}
\usepackage{graphicx}
\usepackage{subcaption}
\usepackage{physics}
\usepackage{tensor}
\usepackage{siunitx}
\usepackage[version=4]{mhchem}
\usepackage{tikz}
\usepackage{xcolor}
\usepackage{listings}
\usepackage{autobreak}
\usepackage[colorlinks,unicode]{hyperref} % , linkcolor=black, anchorcolor=black, citecolor=black, urlcolor=black, filecolor=black
\usepackage{xurl}
\usepackage[most]{tcolorbox}
\usepackage[backend=bibtex,sorting=none,doi=false,isbn=false,url=false]{biblatex}
\addbibresource{monolayer-WTe2.bib}
\addbibresource{methods.bib}

\usepackage{prettyref}

% Page style; to be removed when this article is placed in another template
\geometry{left=3.18cm,right=3.18cm,top=2.54cm,bottom=2.54cm}
\titlespacing{\paragraph}{0pt}{1pt}{10pt}[20pt]
\setlength{\droptitle}{-5em}


% Math operators
\DeclareMathOperator{\timeorder}{\mathcal{T}}
\DeclareMathOperator{\diag}{diag}
\DeclareMathOperator{\legpoly}{P}
\DeclareMathOperator{\primevalue}{P}
\DeclareMathOperator{\sgn}{sgn}
\newcommand*{\ii}{\mathrm{i}}
\newcommand*{\ee}{\mathrm{e}}
\newcommand*{\const}{\mathrm{const}}
\newcommand*{\suchthat}{\quad \text{s.t.} \quad}
\newcommand*{\argmin}{\arg\min}
\newcommand*{\argmax}{\arg\max}
\newcommand*{\normalorder}[1]{: #1 :}
\newcommand*{\pair}[1]{\langle #1 \rangle}
\newcommand*{\fd}[1]{\mathcal{D} #1}
\DeclareMathOperator{\bigO}{\mathcal{O}}


% Embedded codes
\lstdefinestyle{console}{
    basicstyle=\footnotesize\ttfamily,
    breaklines=true,
    postbreak=\mbox{\textcolor{red}{$\hookrightarrow$}\space}
}

% Reference formatting
\newrefformat{fig}{Fig.~\ref{#1}}
\newrefformat{tbl}{Table~\ref{#1}}

% TiKZ settings
\usetikzlibrary{calc}
\tikzset{every picture/.style={line width=0.3pt}} 

% Displaying chemical formula in bookmarkers

\pdfstringdefDisableCommands{%
  \def\\{}%
  \def\ce#1{<#1>}%
}

\pdfstringdefDisableCommands{%
  \def\texttt#1{<#1>}%
  \def\mathbb#1{#1}%
}
\pdfstringdefDisableCommands{\def\eqref#1{(\ref{#1})}}

\makeatletter
\pdfstringdefDisableCommands{\let\HyPsd@CatcodeWarning\@gobble}
\makeatother



\newcommand{\citetime}[1]{item \ref{#1} in \prettyref{sec:timeline}}
\newcommand{\address}[1]{\href{#1}{\url{#1}}}
\newcommand{\shortcode}[1]{\texttt{#1}}

\DeclareSIUnit{\au}{a.u.}

\lstset{style = console}

% Labels 
\newcommand{\strainplot}{\prettyref{fig:one-shot}(d)}
\newcommand{\strainband}{\prettyref{fig:one-shot}(f)}
\newcommand{\nostrainband}{\prettyref{fig:one-shot}(e)}
\newcommand{\overlap}{\prettyref{fig:overlap}}

\title{Supplementary material}
\author{Bowen Hou, Jinyuan Wu, Xingzhi Sun, Smita Krishnaswamy, and Diana Y. Qiu}

\begin{document}

\maketitle

\begin{equation}
    M_{nn'}(\vb{k}, \vb{q}, \vb{G}) = \mel*{n \vb{k} + \vb{q}}{\ee^{\ii (\vb{q} + \vb{G}) \cdot \vb{r}}}{n' \vb{k}}
\end{equation}

\begin{equation}
    \chi_{\vb{G} \vb{G}'}(\vb{q}, \omega = 0)
    = \sum_{\vb{k}} \sum_{n}^{\text{occ}} \sum_{n'}^{\text{emp}} 
    M_{n n'} (\vb{k}, \vb{q}, \vb{G}) M^*_{nn'} (\vb{k}, \vb{q}, \vb{G}') 
    \frac{
        2
    }{
        E_{n \vb{k} + \vb{q}} - E_{n' \vb{k}} 
    }
\end{equation}

In the conventional pseudobands technique (TODO: cite), 
the empty states are divided into one low-energy protected subspace 
and a series of pseudobands blocks; 
each of the blocks contain bands with comparable energies.
The high-energy terms in $\chi_{\vb{G} \vb{G}'}(\vb{q}, \omega = 0)$ are therefore replaced by  
\begin{equation}
    \chi^{\text{P.B. blocks}}_{\vb{G} \vb{G}'}(\vb{q}, \omega = 0)
    = \sum_{\vb{k}}  \sum_{S}^{\text{P.B. terms}} \frac{
        2
    }{
        E_{n \vb{k} + \vb{q}} - \bar{E}_S 
    } 
    \sum_{n_1', n_2'}^{S} \sum_{n}^{\text{occ}}
    M_{n n'_1} (\vb{k}, \vb{q}, \vb{G}) M^*_{nn'_2} (\vb{k}, \vb{q}, \vb{G}') .
    \label{eq:chi-pseudobands}
\end{equation}
Because the energies of states in each pseudobands block are comparable, 
in each pseudobands block, the states share the same set of predominant $\vb{G}$ vectors.
Thus the states in pseudobands block $S$ can be written as 
\begin{equation}
    \braket*{\vb{r}}{n' \vb{k}} =  \sum_i^N 
    c_{n' \vb{k}}^{(i)} \ee^{\ii (\vb{k} + \vb{G}^{(i)}_{S \vb{k}}) \cdot \vb{r}}, \quad 
    \sum_{i}^{N} c^{(i)}_{n'_1 \vb{k}} c^{(i)*}_{n'_2 \vb{k}} = \delta_{n'_1 n'_2},
\end{equation}
and therefore 
\begin{equation}
    M_{nn'}(\vb{k}, \vb{q}, \vb{G}) = \sum_{i}^{N} c_{n' \vb{k}}^{(i)} c_{n \vb{k} + \vb{q}}^*(\vb{G} + \vb{G}_{S \vb{k}}^{(i)}).
\end{equation}
The pseudobands approximation of the diagonal $\chi_{\vb{G} \vb{G}}$ components of the polarizability is proportional to 
\begin{equation}
    \begin{aligned}
    &\quad \sum_n^{\text{occ}} M_{n n'_1} (\vb{k}, \vb{q}, \vb{G}) M^*_{n n'_2} (\vb{k}, \vb{q}, \vb{G}) \\
    &= \sum_{i , j}^N 
    c^{(i)}_{n'_1 \vb{k}} c^{(j) *}_{n'_2 \vb{k}}
    \sum_{n}^{\text{occ}} 
    c_{n \vb{k} + \vb{q}}^*(\vb{G} + \vb{G}^{(i)}_{S \vb{k}})
    c_{n \vb{k} + \vb{q}}  (\vb{G} + \vb{G}^{(j)}_{S \vb{k}}) \\
    &\approx \sum_{i , j}^N  c^{(i)}_{n'_1 \vb{k}} c^{(j) *}_{n'_2 \vb{k}} \delta_{ij} = \delta_{n_1' n_2'}.
    \end{aligned}
\end{equation}
In the third step we argue that the summation over the occupied states leads to an approximate orthogonal relation,
because the subspace spanned by the dominant $\vb{G}$ components of the occupied states 
is expected to largely overlap with the subspace of the occupied states.
Therefore, the unwanted non-diagonal terms in \eqref{eq:chi-pseudobands} can be ignored when $\vb{G} = \vb{G}'$.
The 

which justifies the validity of the pseudobands technique for high-energy bands.



\end{document}